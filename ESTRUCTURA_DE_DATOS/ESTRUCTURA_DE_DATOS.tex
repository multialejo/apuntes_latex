\documentclass[12pt]{article}
\usepackage[utf8]{inputenc} % Codificación
\usepackage[spanish]{babel} % Lenguaje
\usepackage[T1]{fontenc}
\usepackage{mathpazo} % tipo de letra
\usepackage[left=1.5cm, right=1.5cm, top=1.5cm, bottom=1.5cm]{geometry} % Diseño de la hoja
\usepackage[document]{ragged2e} %alinear
\usepackage{mathrsfs}
%\usepackage{upgreek}
\usepackage{amssymb}
\usepackage{mathtools}
\usepackage{fancyhdr}
\usepackage{parskip}

%CODIGO
\usepackage{listings} % Para código fuente
\usepackage{listingsutf8}

\usepackage{ latexsym }
\usepackage{graphicx}
\usepackage{float}
\usepackage{pgfplots}
\usepackage{subcaption}
\usepackage{pgf-pie}
\usepackage{tikz}
\pgfplotsset{compat=1.17}
\usepackage{pgfplotstable}

\usepackage[dvipsnames]{xcolor} %resaltar con color, utilizar \colorbox{BurntOrange}{texto texto texto} link:https://es.overleaf.com/learn/latex/Using_colours_in_LaTeX (colores)
\pagestyle{fancy}

\lhead[]{\textbf{Escuela Politécnica Nacional}}
\chead[]{Calculo Diferencial}
\rhead[]{\textbf{Nombre:} Jesus Yanchaliquin}

\lfoot[]{Bartle}
\cfoot[]{\rightmark}
\rfoot[]{\textbf{Pagina: }\thepage}
\renewcommand{\footrulewidth}{0.09pt}

%FORMATO DEL CODIGO

% Definir estilo personalizado para C++
\definecolor{codegreen}{rgb}{0,0.6,0}
\definecolor{codegray}{rgb}{0.5,0.5,0.5}
\definecolor{codepurple}{rgb}{0.58,0,0.82}
\definecolor{backcolour}{rgb}{0.95,0.95,0.92}

\lstdefinestyle{mystyle}{
    backgroundcolor=\color{backcolour},   
    commentstyle=\color{codegreen},
    keywordstyle=\color{magenta},
    numberstyle=\tiny\color{codegray},
    stringstyle=\color{codepurple},
    basicstyle=\ttfamily\footnotesize,
    breakatwhitespace=false,         
    breaklines=true,                 
    captionpos=b,                    
    keepspaces=true,                 
    numbers=left,                    
    numbersep=5pt,                  
    showspaces=false,                
    showstringspaces=false,
    showtabs=false,                  
    tabsize=2
}

\lstset{style=mystyle}




\begin{document}
%-----------------------------------------------------------
\begin{titlepage}
\centering
{\bfseries\LARGE Escuela Politécnica Nacional \par}
\vspace{1cm}
{\scshape\Large Facultad de Ciencias \par}
\vspace{3cm}
{\scshape\Huge ESTRUCTURA DE DATOS \par}
\vspace{3cm}
{\itshape\Large Apuntes de la materia de \textbf{ESTRUCTURAS DE DATOS}, con los problemas resueltos \par}
\vfill
{\Large Autor: \par}
{\Large Jesus Padilla \par}
{\Large Alejandro Calderon \par}
\vfill
{\Large Septiembre 2023 \par}

\end{titlepage}
%-----------------------------------------------------------
\tableofcontents
\newpage
\vspace{\baselineskip} % No se para que sirve pero dejale ahi jajajajaja

\section{Algoritmos y complejidad computacional}
{

    \begin{enumerate}
        \item[$\bullet$]\textbf{\textcolor{blue}{(Ejercicio 1.1): } Dado un conjunto $X = \{x_1, \dots, x_n\}$ de variables booleanas y $T = \{x_i, \neg x_i : x_i \in X\}$ el conjunto de literales.  
        El problema SAT consiste en una familia de $m$ cláusulas $C = \{C_1, C_2, \dots, C_m\}$, donde cada cláusula $i$ contiene $m_i$ literales.  
        La familia $C$ de cláusulas sobre $X$ tiene una respuesta de verdadero si y solo si en todas las cláusulas al menos uno de sus miembros es verdadero.}
        
        \textbf{Por ejemplo: }

            \[
                C = (C_1, C_2, C_3) = ((x_1 \lor x_3 \lor \neg x_4 \lor x_5), (x_2 \lor \neg x_3 \lor x_4), (x_1 \lor x_2 \lor x_3))
            \]

        \textbf{Claramente, una solución es: $(x_1, x_2, x_3, x_4) = (V, V, F, F, V)$ Determine la longitud de codificación de una instancia del problema anterior.}

        \textbf{Solución:}

        {
            Para determinar la \textbf{longitud de codificación} de una instancia del problema SAT, seguimos el modelo teórico:

            \begin{enumerate}
                \item \textbf{Número de variables:} n = 5 ($x_1, x_2, x_3, x_4, x_5$).
                \item \textbf{Número de cláusulas:} $m = 3$.
                \item \textbf{Número total de literales en las cláusulas:}
                \begin{itemize}
                    \item $C_1$ tiene 4 literales.
                    \item $C_2$ tiene 3 literales.
                    \item $C_3$ tiene 3 literales.
                    \item \textbf{Total:} $4 + 3 + 3 = 10$.
                \end{itemize}
            \end{enumerate}
            
            Dado que cada variable puede tomar \textbf{2 valores} (Verdadero $V$ o Falso $F$), lo que equivale a \textbf{1 bit por variable}, y para almacenar cada literal debemos indicar si está negado ($\neg x_i$), usamos \textbf{1 bit adicional por literal}.
            
            La longitud de codificación se calcula como:
            
                \[
                    L = n + \sum_{i=1}^{m} m_i
                \]
            
            Sustituyendo los valores:
            
                \[
                    L = 5 + 10 = 15 \text{ bits}
                \]
            
            \textbf{Por lo tanto, la longitud de codificación de esta instancia del problema SAT es 15 bits.}
        }
        
        \item[$\bullet$]\textbf{\textcolor{blue}{(Ejercicio 1.2): } Un palíndromo es una palabra o frase que se lee igual de adelante hacia atrás que de atrás hacia adelante.  
        Por ejemplo: \textit{reconocer}, \textit{anilina}, \textit{radar}.}
        
        \textbf{Si una frase puede ser considerada como un vector de caracteres:}  
        
            \[
                x = (a, n, i, l, i, n, a)
            \]
        
        \textbf{Escribir e implementar un \textbf{algoritmo en C++} que permita determinar si un vector almacena o no un palíndromo.  
        Determinar el orden del algoritmo construido. ¿El algoritmo es polinomial?}
        
        \textbf{Solución: }
        
        {
            Para determinar si un vector es un palíndromo, comparamos los caracteres desde ambos extremos hacia el centro.

            \begin{lstlisting}[language=C++]
#include <iostream>
#include <vector>

using namespace std;

// Funciin para verificar si un vector de caracteres es palindromo
bool esPalindromo(const vector<char>& palabra) {
    int n = palabra.size();
    for (int i = 0; i < n / 2; i++) {
        if (palabra[i] != palabra[n - 1 - i]) {
            return false; // No es palindromo
        }
    }
    return true; // Es palindromo
}

int main() {
    vector<char> palabra = {'a', 'n', 'i', 'l', 'i', 'n', 'a'}; // Caso de prueba

    if (esPalindromo(palabra)) {
        cout << "El vector almacena un palindromo." << endl;
    } else {
        cout << "El vector NO almacena un palindromo." << endl;
    }

    return 0;
}      
            \end{lstlisting}

            El algoritmo compara los elementos extremos hasta la mitad del vector.

            \begin{itemize}
                \item \textbf{Tiempo de ejecución:} \( O(n) \) (recorre la mitad del vector).
                \item \textbf{Espacio adicional usado:} \( O(1) \) (solo variables auxiliares).
            \end{itemize}

            \textbf{Conclusión:}  
            El algoritmo es \textbf{polinomial}, ya que su tiempo de ejecución es \( O(n) \), lo cual es eficiente.
        }
        
        \item[$\bullet$]\textbf{\textcolor{blue}{()}  }{}
        
        \item[$\bullet$]\textbf{\textcolor{blue}{()}  }{}
        
        \item[$\bullet$]\textbf{\textcolor{blue}{()}  }{}
        
        \item[$\bullet$]\textbf{\textcolor{blue}{()}  }{}
        
        \item[$\bullet$]\textbf{\textcolor{blue}{()}  }{}
        
        \item[$\bullet$]\textbf{\textcolor{blue}{()}  }{}
        
        \item[$\bullet$]\textbf{\textcolor{blue}{()}  }{}
    \end{enumerate}
    

    
}

\section{Algoritmos de Ordenamiento}
{
    
    \begin{enumerate}
        \item[$\bullet$]\textbf{\textcolor{blue}{()}  }{}
        
        \item[$\bullet$]\textbf{\textcolor{blue}{()}  }{}
        
        \item[$\bullet$]\textbf{\textcolor{blue}{()}  }{}
        
        \item[$\bullet$]\textbf{\textcolor{blue}{()}  }{}
        
        \item[$\bullet$]\textbf{\textcolor{blue}{()}  }{}
        
        \item[$\bullet$]\textbf{\textcolor{blue}{()}  }{}
        
        \item[$\bullet$]\textbf{\textcolor{blue}{()}  }{}
        
        \item[$\bullet$]\textbf{\textcolor{blue}{()}  }{}
        
        \item[$\bullet$]\textbf{\textcolor{blue}{()}  }{}
        
        \item[$\bullet$]\textbf{\textcolor{blue}{()}  }{}
    \end{enumerate}
    
}

\section{Conceptos Básicos de C++}
{
    
    \begin{enumerate}
        \item[$\bullet$]\textbf{\textcolor{blue}{()}  }{}
        
        \item[$\bullet$]\textbf{\textcolor{blue}{()}  }{}
        
        \item[$\bullet$]\textbf{\textcolor{blue}{()}  }{}
        
        \item[$\bullet$]\textbf{\textcolor{blue}{()}  }{}
        
        \item[$\bullet$]\textbf{\textcolor{blue}{()}  }{}
        
        \item[$\bullet$]\textbf{\textcolor{blue}{()}  }{}
        
        \item[$\bullet$]\textbf{\textcolor{blue}{()}  }{}
        
        \item[$\bullet$]\textbf{\textcolor{blue}{()}  }{}
        
        \item[$\bullet$]\textbf{\textcolor{blue}{()}  }{}
        
        \item[$\bullet$]\textbf{\textcolor{blue}{()}  }{}
    \end{enumerate}
    
}

\section{Estructura de Datos Básicas}
{
    
    \begin{enumerate}
        \item[$\bullet$]\textbf{\textcolor{blue}{()}  }{}
        
        \item[$\bullet$]\textbf{\textcolor{blue}{()}  }{}
        
        \item[$\bullet$]\textbf{\textcolor{blue}{()}  }{}
        
        \item[$\bullet$]\textbf{\textcolor{blue}{()}  }{}
        
        \item[$\bullet$]\textbf{\textcolor{blue}{()}  }{}
        
        \item[$\bullet$]\textbf{\textcolor{blue}{()}  }{}
        
        \item[$\bullet$]\textbf{\textcolor{blue}{()}  }{}
        
        \item[$\bullet$]\textbf{\textcolor{blue}{()}  }{}
        
        \item[$\bullet$]\textbf{\textcolor{blue}{()}  }{}
        
        \item[$\bullet$]\textbf{\textcolor{blue}{()}  }{}
    \end{enumerate}
    
}

\section{Arboles Binarios de Búsqueda}
{
    
    \begin{enumerate}
        \item[$\bullet$]\textbf{\textcolor{blue}{()}  }{}
        
        \item[$\bullet$]\textbf{\textcolor{blue}{()}  }{}
        
        \item[$\bullet$]\textbf{\textcolor{blue}{()}  }{}
        
        \item[$\bullet$]\textbf{\textcolor{blue}{()}  }{}
        
        \item[$\bullet$]\textbf{\textcolor{blue}{()}  }{}
        
        \item[$\bullet$]\textbf{\textcolor{blue}{()}  }{}
        
        \item[$\bullet$]\textbf{\textcolor{blue}{()}  }{}
        
        \item[$\bullet$]\textbf{\textcolor{blue}{()}  }{}
        
        \item[$\bullet$]\textbf{\textcolor{blue}{()}  }{}
        
        \item[$\bullet$]\textbf{\textcolor{blue}{()}  }{}
    \end{enumerate}
    
}

\section{Grafos y su Implementación computacional}
{
    
    \begin{enumerate}
        \item[$\bullet$]\textbf{\textcolor{blue}{()}  }{}
        
        \item[$\bullet$]\textbf{\textcolor{blue}{()}  }{}
        
        \item[$\bullet$]\textbf{\textcolor{blue}{()}  }{}
        
        \item[$\bullet$]\textbf{\textcolor{blue}{()}  }{}
        
        \item[$\bullet$]\textbf{\textcolor{blue}{()}  }{}
        
        \item[$\bullet$]\textbf{\textcolor{blue}{()}  }{}
        
        \item[$\bullet$]\textbf{\textcolor{blue}{()}  }{}
        
        \item[$\bullet$]\textbf{\textcolor{blue}{()}  }{}
        
        \item[$\bullet$]\textbf{\textcolor{blue}{()}  }{}
        
        \item[$\bullet$]\textbf{\textcolor{blue}{()}  }{}
    \end{enumerate}

% \item[$\bullet$]\textbf{\textcolor{blue}{()}  }{}    
}

\end{document}